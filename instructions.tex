%---------------------------------------------------------------------------
% Lines 4 to 28 below define the style and layout of your document and should not be edited.

\documentclass[10pt]{article}
\usepackage{graphicx, amsmath, amssymb, amsfonts, amsthm, authblk, fullpage, listings, color, hyperref}
\usepackage[english]{babel}

\definecolor{mygreen}{rgb}{0,0.6,0}
\definecolor{mygray}{rgb}{0.85,0.85,0.85}
\definecolor{mymauve}{rgb}{0.58,0,0.82}

\lstset{ %
	basicstyle=\footnotesize\ttfamily,	% sets the size and font of code characters
	breaklines=true,					% automatic line breaking only at whitespace
	captionpos=b,						% sets the caption-position to bottom
	numbers=left,						% sets line numbers on the left
	tabsize=4,							% sets tabs equal to four spaces
	showstringspaces=false,				% do not have little characters for spaces in strings
	backgroundcolor=\color{mygray},   	% set background color
	stringstyle=\color{mymauve},     	% set colour of strings in the code
	keywordstyle=\color{blue},       	% set colour of keywords
	commentstyle=\color{mygreen},    	% set colour of comments in the code
}

\newcommand{\py}{\lstinline[language=Python]}
\def\UrlFont{\em}

\setlength{\parindent}{0em}
\setlength{\parskip}{0.5em}


%---------------------------------------------------------------------------
% Lines 33 to 41 below set up the title of the document. Lines 33 and 34 can be edited to define the title and author names.
\title{Enter Title of Your Coursework Here}
\author{Enter Names and Student Numbers Here}
\affil{
	MA2760: Mathematical Investigations with Python\\
	School of Mathematics, \\
	Cardiff University. 
}
\begin{document}
\maketitle


%---------------------------------------------------------------------------
% The remainder of this document should be edited for your coursework.
\section{Introduction}

This document gives some details on how reports for this module should be typeset. Please refer to this document's source code in the file \texttt{instructions.tex} for further information on how this document has been written using \LaTeX. 

Perhaps the easiest way to compile \texttt{instructions.tex} is to simply copy it into a project on the online \LaTeX\ editor Overleaf (\url{https://www.overleaf.com/}). Note that you will also need to copy across the file \texttt{examplefig.pdf} to this project too. 


%---------------------------------------------------------------------------
\section{Details on Section Numbering}

In this document, sections, subsections, and sub-subsections should be numbered consistently, as the following demonstrates.

\subsection{This is a Subsection}

This is a subsection.

\subsubsection{This is a Sub-Subsection}

This is a sub-subsection.


%---------------------------------------------------------------------------
\section{Inserting Lists}

Lists in \LaTeX\ can be defined using, amongst other things, the \texttt{enumerate} and \texttt{itemize} environments. Here is an example of the latter:

\begin{itemize}
\item Apples,
\item Oranges,
\end{itemize}

and here is an example of the former:

\begin{enumerate}
\item Python,
\item C++,
\item Visual Basic.
\end{enumerate}


%---------------------------------------------------------------------------
\section{Inserting Figures}

Figures can be introduced into a document using the command \verb|\includegraphics|. It is desirable to label each figure and provide a caption. This can be done using the \texttt{figure} environment. An example is provided in Figure~\ref{fig:graph}.

\begin{figure}
\centering
\includegraphics[scale=0.5]{examplefig.pdf} 
\caption{An example graph with twenty vertices.}
\label{fig:graph}
\end{figure}

Note that various different image formats can be used with the \verb|\includegraphics| command, though \textbf{.pdf} and \textbf{.gif} files will often give the best resolution.


%---------------------------------------------------------------------------
\section{Inserting Code}

To add snippets of code to a document, you should use the \texttt{lstlisting} environment, specifying within the command that the Python programming language is being used. Note that in this environment the text is presented exactly as it has been typed in; it is therefore best to paste your final code directly from your Python editor. Here is an example:
   
\begin{lstlisting}[language = python, frame=single]
total = 0.0
i = 0

while i < 10:
	x = float(input("Enter number >> "))
	total += x
	i += 1

print("Average = " + str(total / float(i)))
\end{lstlisting}

You can also write pieces of code in-line using the \verb|\py| command. For example, a list of 100 zeros can be declared by the command \py{X = [0 for i in range(100)]}

Output from a program can also be specified using the \texttt{lstlisting} environment. In this case we do not require syntax highlighting, so there is no need to specify a programming language in \texttt{lstlisting} command in \LaTeX. Here is some example output from the above program.

\begin{lstlisting}[frame=single]
Enter number >> 1
Enter number >> 2
Enter number >> 3
Enter number >> 4
Enter number >> 5
Enter number >> 6
Enter number >> 7
Enter number >> 8
Enter number >> 9
Enter number >> 10
Average = 5.5
\end{lstlisting}


%---------------------------------------------------------------------------
\section{Inserting Tables}
Tables are written in \LaTeX\ using the \texttt{table} and \texttt{tabular} commands. As with figures, these should be labeled and captioned correctly and referred to in the text. Table~\ref{tab:example} gives a small example.

\begin{table}
\centering
\begin{tabular}{llr}
\hline\hline
Country		& Capital City	& Population\\
\hline
Wales		& Cardiff		& 3,009,000		\\
Scotland	& Edinburgh		& 5,373,000		\\
England		& London		& 54,790,000 	\\
Netherlands & Amsterdam & 17,180,000	\\
\hline
\end{tabular}
\caption{Example table showing data on four countries.}
\label{tab:example}
\end{table}

%---------------------------------------------------------------------------
\section{Inserting Links}
If you want to put a web link into your document, please use the \verb|\url| command. When the pdf file is generated, this will create a hyper-link and will also format the url correctly. For example: \url{https://nation.cymru/}. 

%---------------------------------------------------------------------------
\section{Further Information}
\LaTeX\ is all over the Internet, and a quick search will often answer your queries. In addition to any previous module notes you have been given on \LaTeX, a good guide can be found at \url{https://en.wikibooks.org/wiki/LaTeX}. 

As mentioned, a convenient place to write and compile \LaTeX\ is within a ``cloud-based'' environment. This is probably the most flexible option, as there is no need to install anything, and it should work on any computer with a web browser. A good option is \url{https://www.overleaf.com}, though others are available. If you prefer you can also install \LaTeX\ on your own computer for free. See, for example, MikTeX \url{http://miktex.org/} for Windows, Texlive \url{https://www.tug.org/texlive/} for Linux, and MacTeX \url{https://tug.org/mactex/} for Macs.

%---------------------------------------------------------------------------
\end{document}
